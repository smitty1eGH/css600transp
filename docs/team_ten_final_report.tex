\documentclass[12pt,letterpaper]{article}


\usepackage{hyperref}
\usepackage{amsmath}
\usepackage{listings}


\begin{document}
\title{Evac Sim: Fall 2020 CSS600 }

\author{Justin Downes and Chris Smith}
\date{December 2020}
\maketitle

\begin{abstract}
blah blah blah

\end{abstract}
\section {Introduction}



\section {Background}
\subsection{Previous Work}
\cite{almeidaCrowdSimulationModeling2013} 
\cite{kneidl}
\cite{kuligowskil}
\cite{abmEvac}
\cite{zhouSimulationPedestrianEvacuation2019}

This paper is key as it is extremely similar and a netlogo implementation.  we should know this paper and incorporate into our paper \cite{prioritEvac}

\section {Methodology}
this introductory section will discuss how we approached the problem
\subsection{Environment}
this section will discuss the netlogo environment, how maps are loaded, the different patch types, the different agent speeds

need to talk about patch parameters
\subsection{Movement Mechanisms}
this section will discuss agent and fire movement algorithms

The cost algorithm, where $P_s$ is a safety patch, where $A_w$ is the person path weight, a weight that a person adds to a patch due to blocking (this is configurable by the user)
\begin{align}
cost(P)  = distance(P_s, P) + Agent(P) * A_w \nonumber \\
Agent(P)=
\begin{cases}
1, & \text{Agent Present}  \\
0, & \text{Agent Not Present} 
\end{cases}
\end{align}

this can be configured through equal diagonal weight flag. if False the Distance is the Manhattan distance \footnote{ https://www.sciencedirect.com/topics/mathematics/manhattan-distance}. If true the distance is the Chebyshev Distance \footnote{https://en.wikipedia.org/wiki/Chebyshev\_distance}
\begin{align}
distance(P_1, P_2)  = \nonumber\\
equal-diagonal-weight?=
\begin{cases}
	max(|x_1-x_2|, |y_1-y_2|), & True \\
	|x_1-x_2|+ |y_1-y_2|, & False
\end{cases}
\end{align}


patch to move to algorithm, where $A_p$ is the patch for a given agent, $P_x$ \& $P_y$ are the $X$ \& $Y$ coordinates for a given patch
\begin{align}
neighbors_4 (P)  = \{patch(P_x - 1, P_y -1), patch(P_x + 1, P_y -1), \nonumber \\ 
patch(P_x - 1, P_y + 1),patch(P_x + 1, P_y + 1)\}   
\end{align}

\begin{equation}
move(A) = min(cost(neighbors_4 (A_p)))
\end{equation}

Person will wait for a better patch.  This is configurable by the user. If it is on then a person will wait for a patch that is less cost than it's current
\begin{equation}
people-wait?=
\begin{cases}
min(cost(A_p), move(A)), & True\\
move(A), & False
\end{cases}
\end{equation}

add person spacing algorithm, people try to avoid each other, configurable through the $add-person-spacing?$ flag and the $person\_path\_weight$, $A_w$, parameter.  here, $A_w$, is scaled by a factor of 10 since it is not the weight of being in the same square as another but of being next to another person.

\begin{equation}
add-person-spacing?=
\begin{cases}
	cost(P) \sum Agent(neighbors_4(P)) * A_w / 10, & True \\
	cost(P), & False
\end{cases}
\end{equation}

\cite{mirahmadiNovelAlgorithmRealtime2012}
\footnote{http://www.cs.us.es/~fsancho/?e=131}

\subsection{Experiments}

describe our experiment harness that we built, netlogo behavior space, sql alchemy

\subsubsection{experiments that are based on layouts.}  

-exit dims maps:  see if placement has an effect on mean and max escape time.  line plot for each number connecting a,b, and c points where each point is the average of multiple (10) runs . separate plots for mean and max

- choke point escape: same as above

- need to run experiment on using a.map that plots mean escape time as we increase number of agents.  probably 50 to 500 step size 10. we would expect to see this scale linearly but if not then there may be an interesting story.

\subsubsection{experiments that are based on agent mechanisms}

- vary experiments of person path weighting and agent speeds/distributions

\subsubsection{experiments that are based on patch parameters.}

-  this may be obe by the exclusion of the fire mechanism

\subsubsection{experiments that replicate emergent crowd behaviors from other papers}
- maiinly from this paper \cite{almeidaCrowdSimulationModeling2013}  .  

- herding and flocking

-arching and crowding

if we can show that we achieve similar results even though we use a simplified pathing algorithm and abm environment i think that would be insightful


\section{Results}



\section {Conclusion}

\bibliographystyle{plain}
\bibliography{css600}

\end{document}
